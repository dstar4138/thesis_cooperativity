\documentclass[11pt]{artikel3}
\usepackage{fullpage, setspace, graphicx}
\usepackage[margin=1in]{geometry}
\usepackage{times}
\usepackage{enumitem}

\title{RIT Department of Computer Science\\MSc Thesis Pre-Proposal:\\
	\emph{Process Cooperativity as a Feedback Metric in Message-Passing Concurrent Languages}
}
\author{Alexander Dean, Chair: Matthew Fluet}
\date{\today}

\begin{document}
\maketitle

\section{Problem Description}

Runtime systems for concurrent languages have begun to utilize feedback mechanisms to influence their
scheduling behaviour as the application proceeds. These feedback mechanisms rely on metrics by which to
grade any alterations made to the scheduling system. As the application's phase shifts, the feedback mechanism
is tasked with modifying the scheduler to reduce it's overhead and increase the application's efficiency.

For example, feedback mechanism could utilize the ratio of I/O to CPU bound processes within a set number
of rounds to be the metric by which it grades it's efficiency. If it is too high the runtime should give
each process less time to compute, so it can get through more processes in a round, thus reducing the drag
the CPU bound processes have on the system. The inverse would have the opposite effect.

Cooperativity is another possible metric by which to grade a system. In biochemistry the term cooperativity
is defined as the increase or decrease in the rate of interaction between a reactant and a protein as
the reactant concentration increases. This definition translates well as an information theoretic definition
as: the increase or decrease in the rate of interaction between a process and a communication mode as the
number of processes increase.

%Communication mode in this sense would be a function of interprocess communication such as
%asynchronous mail-boxes, synchronous channel or even shared memory. In any example the sending process will
%need to interact with this method as a means of interacting with a concurrent process.

I will attempt to answer the following questions:
\begin{itemize}[leftmargin=.5in]
	\item Are the abstractions of a process and communication channel conducive to quantifying
		  cooperativity of a system? What generalizations can be made without obscuring important
		  factors?
	\item What factors effect the cooperativity of a runtime system? Also, what mechanisms can be used
		  to update the scheduling system to positively improve cooperativity in a given phase?
	\item Is cooperativity a comparable metric for feedback scheduling?
\end{itemize}

\section{Related Work}

Scheduling based on feedback from the running system is not new \cite{dietz1997use}. There have even been multiple
metrics explored, such as heap size \cite{white2012automated}, process locality \cite{debattista2002cache}, and interactivity \cite{reppy1993concurrent}.
In \cite{ritson2012multicore} they mention communication efficiency as a rational for their implementation of a runtime scheduler using process locality
as the primary metric. However, this differs from capturing the cooperativity in that, while communication efficiency is important, it misses the quantity
of communication which could indicate needed changes in scheduling quantum.

\section{Motivation for Research}

The motivation is twofold. Current feedback scheduling implementations fail to take process-cooperativity
into account when applying feedback. A scheduler implementation and metric calculation would aid in future
feedback schedulers. However, this hints at other missing metrics that could be useful for adapting to
application phases. A closer evaluation of current scheduling techniques on common language primitives
will allow for a constructive comparative analysis. This, as far as I'm aware, has not been done at the injected
runtime level but rather only at the OS or Application levels with job schedulers.

\section{Methodology}

I have already begun to work on a compiler built with swappable scheduler's in mind. The language
is a simple lambda and process calculi with synchronous swap channels so as to minimize the language
characteristics which may mask the process-based metrics, such as differentiating between CPU, I/O,
and Channel bound processes. This compiler will also allow me to directly test multiple metrics and
scheduling techniques and visualize the system phases at any given point. I will attempt to do a
comparative analysis of several popular scheduling techniques against a new algorithm I will need to design
which uses cooperativity. This, in turn, will evaluate the effectiveness of the process and communication
channel abstractions utilized in the language.

\section{Expected Outcomes}

A new tool for comparative analysis of some current runtime schedulers will be available as well as a
set of high level language abstractions for future runtime scheduler testing.

Using this analysis I expect to also contribute an original scheduling algorithm which takes cooperativity
of processes into account within it's feedback mechanism. This will also require an examination of fairness
guarantees of both the communication mode and process selection.

\nocite{*}
\bibliographystyle{plain}
\bibliography{preproposal}

\end{document}
