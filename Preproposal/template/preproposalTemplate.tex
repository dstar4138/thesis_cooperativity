\documentclass[11pt]{artikel3}
\usepackage{fullpage, setspace, graphicx}
\usepackage[margin=1in]{geometry}
\usepackage{times}

\title{RIT Department of Computer Science\\MSc Thesis Pre-Proposal:\\\emph{Interactivity of Communicating Processes}}
\author{Alexander Dean}
\date{\today}

\begin{document}
\maketitle

The sections shown below are adapted from the topic analysis forms provided in ``Writing the Doctoral Dissertation" (2nd edition) by Davis and Parker (pages 82-88). Your final document should be 1-2 pages including references. {\bf The final pre-proposal may present the items below in any format, but using prose (not bulleted lists).} 

\section{Problem}
Identify what problem you are addressing, both in terms of the research area, and the \emph{specific} problem that you will be working on:
\begin{itemize}
	\item For a thesis, a hypothesis (`thesis statement') that you will test in your research.
	\item For a project, identify the work required (e.g. implementation and/or experiment) that needs to be completed. If you are completing a project, make sure to speak with your advisor about the expected deliverables; this will include a written project report.
\end{itemize}

{\bf Motivate your problem.} 
\begin{itemize}
	\item What is the significance of your problem? 
	\item What applications or new opportunities will solving your problem provide?
\end{itemize}

{\bf Related Work.}
%Demonstrate the connection between your chosen problem and its foundations in existing work. 
\begin{itemize}
	\item What are the key theoretical models (e.g. process-based, formal language/complexity models, probability-based) and algorithms have been applied toward this problem previously? 
	\item What limitation and/or opportunity do you plan to address in your project/thesis?
	\item
In the related research literature, how is success measured (e.g. metrics and/or coverage of problem aspects)?
\end{itemize}

\section{Methodology}

\begin{itemize}
\item For projects, which libraries or software tools will be used for development, and at the highest level, what is the software design?
\item For theory-based projects and theses, what are the key theorems to be developed and/or proven? What proof techniques will be used?
\item For other theses, what algorithms will be adapted or devised, and what algorithms will they be compared with in your experiments?
\end{itemize}

\section{Evaluation}

\begin{itemize}
\item What data and software will be needed for your evaluation? 
\item What metrics
will you use to measure success? Commonly these include some subset of time, space, and accuracy (recognition rate, precision, recall, etc.).
Almost always, this should include reference to the evaluation methods described in the related work.
\item For empirical theses, an experiment is needed: which algorithms will be compared, and using what sets of parameters for each? If people are involved in the experiments, how will the experiment control for unwanted bias or confounds? {\bf How does the experiment test the hypothesis?}
\item {\bf How will you know when you are done?}
\end{itemize}

\section{Evaluation Outcomes}

\begin{itemize}
\item For your chosen assessment methods, what are the possible outcomes? 
\item Under which outcomes are the project goals achieved, or the hypothesis confirmed or rejected?
\end{itemize}

\bibliographystyle{plain}
\bibliography{preproposal}

**(Omitted) As an exercise, modify this document to include the references in the {\tt plain.bib} file.

\end{document}
