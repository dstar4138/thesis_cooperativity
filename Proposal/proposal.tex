%%%%%%%%%%%%%%%%%%%%%%%%%%%%%%%%%%%%%%%%%%%%%%%%%%%%%%%%%%%%%%%%%%%%%%%%%%%%%%
%
% PROJECT PROPOSAL  DESCRIPTION:
%   A concise description of the main concepts of the proposed project.
%
% RESEARCH:
%   A list of research activities which led to this project.
%
% EXPERIMENTS:
%   A list of the experiments performed which supported the research.
%
%%%%%%%%%%%%%%%%%%%%%%%%%%%%%%%%%%%%%%%%%%%%%%%%%%%%%%%%%%%%%%%%%%%%%%%%%%%%%%%
% Define a single space environment (copied from doublespace.sty)
% e.g. \begin{singlespace}
%         single-spaced text
%      \end{singlespace}

\documentclass[12pt,american]{article}
\usepackage{fullpage}
\usepackage{bbm}
\usepackage{url}
\usepackage{subfigure}
\usepackage{babel}
\usepackage{times}
\usepackage{graphicx}
\usepackage{amssymb}
\usepackage{lscape}
\usepackage{verbatim}
\usepackage{enumerate}
\usepackage{afterpage}
\usepackage{setspace}


\begin{document}
\thispagestyle{empty} 
\begin{center}
{\em MS Thesis Proposal}\\
\vspace{.5in}
{\large \bf Process Cooperativity as a Feedback Metric \\
            in Concurrent Message-Passing Languages }\\
\vspace{.5in}
{\bf Alexander Dean}\\
\vfill
\
{\em Committee Chair:} Matthew Fluet\\
\vspace{0.1in}
{\em Reader: } The Second Dude\\
 \vspace{0.1in}
Department of Computer Science\\
B. Thomas Golisano College of Computing and Information Sciences \\
Rochester Institute of Technology \\
Rochester, New York \\ [0.3in]
\vspace{0.5in}
\today{}\\
\end{center}
\vfill

%%%%%%%%%%%%%%%%%%%%%%%%%%%%%%%%%%%%%%%%%%%%%%%%%%%%%%%%%%%%%%%%%%%%%%%%%%%%%%%
%%  Collection of useful abbreviations.
\newcommand{\etc} {\emph{etc.\/}}
\newcommand{\etal}{\emph{et~al.\/}}
\newcommand{\eg}  {\emph{e.g.\/}}
\newcommand{\ie}  {\emph{i.e.\/}}
%%%%%%%%%%%%%%%%%%%%%%%%%%%%%%%%%%%%%%%%%%%%%%%%%%%%%%%%%%%%%%%%%%%%%%%%%%%%%%%


%%%%%%%%%%%%%%%%%%%%%%%%%%%%%%%%%%%%%%%%%%%%%%%%%%%%%%%%%%%%%%%%%%%%%%%%%%%%%%%
% Abstract
\section*{Abstract}
This should be a short description of the work and the results: a paragraph 
or two summarizing your project proposal. Note that an abstract is meant to be read 
independently from the rest of the project report so you cannot cite your 
paper or other papers in it. It would be useful to examine other abstracts 
in the many papers you have read to understand what an abstract really is.
%%%%%%%%%%%%%%%%%%%%%%%%%%%%%%%%%%%%%%%%%%%%%%%%%%%%%%%%%%%%%%%%%%%%%%%%%%%%%%%
\vfill{}

%%%%%%%%%%%%%%%%%%%%%%%%%%%%%%%%%%%%%%%%%%%%%%%%%%%%%%%%%%%%%%%%%%%%%%%%%%%%%%%
% This is where the main body of the capstone proposal starts
\setcounter{page}{0} 
\newpage{}

\section{Introduction}
This part of the proposal should be a couple of paragraphs that
describe the reason for your proposal and your project/thesis area at
high level.

\section{Background}
This section should be sufficient for the reader to understand the
project area and the relevance of your efforts in the world of
computer science. The description here should be provide the
motivation to the reader that you are exploring a problem area that is
relevant to the CS community.


\section{Related Work}
Describe what work others have already done in this area. You do need
several citations, and this is how you cite a book by
Silberschatz~\cite{Silberschatz05-text} or a paper by
Dumont~\cite{Dumont2007-robots}.

\section{Hypothesis}
Summarize what you think the problem is, and what your hypothesis
is. Here is a small example based on a successful project by Priyanka
Sinha: ''Using one technique for schema matching does not seem
adequate. The hypothesis underlying this sproject is that a holistic
approach to schema matching based on the three techniques described
earlier would do an effective approach to schema matching.''

Additional description to circumscribe the work so that the reader
knows what you plan to do to establish your hypothesis.

\section{Solution Design and Implementation}
Describe how you plan to design and implement a solution. 

You must also describe how you would use your solution to establish the validity of your hypothesis. Explain the measurements you plan to conduct and how these would establish the validity (or invalidity) of your hypothesis.

\section{Roadmap}

Based on the layout of the 10-week Summer session of 2138, where the tenth week
is the defense. It will primarily be a top-heavy load that will shift as needed
with the inevitability of roadblocks:\\

    April\\
        $4/25$ - Submission of Proposal with timeline.\\

    May\\
        $5/25$ - $5/31$ - Finish base ErLam Compiler and Plugin Scheduler Interface \\

    June \\
        $6/01$ - $6/07$ - Port CML scheduler to Erlang $*$\\
        $6/08$ - $6/14$ - Cooperative Algorithm Development/Implementation \\
        $6/15$ - $6/21$  \\
        $6/22$ - $6/28$ - Implement Test Cases for Scheduler Comparisons\\

    July\\
        $6/29$ - $7/05$ (Schedule Defense)\\
        $7/06$ - $7/12$ - Run tests and compile results\\
        $7/13$ - $7/19$ \\
        $7/20$ - $7/26$ - Draft of Thesis report submission\\

    August\\
        $7/27$ - $8/02$ - Thesis Defense\\
        $8/03$ - $8/09$ -   (Backup Defense dates)\\


$*$ Strech goal to implement a batching Occam-Pi scheduler too. \\

Every week will contain at least one meeting with my chair and every two weeks
must result in an update to my website laying out my progress.

%%%%%%%%%%%%%%%%%%%%%%%%%%%%%%%%%%%%%%%%%%%%%%%%%%%%%%%%%%%%%%%%%%%%%%%%%%%%%%%

%%%%%%%%%%%%%%%%%%%%%%%%%%%%%%%%%%%%%%%%%%%%%%%%%%%%%%%%%%%%%%%%%%%%%%%%%%%%%%%
\bibliographystyle{plain}
% Single space the bibliography to save space.
\singlespacing
\bibliography{proposal}
%%%%%%%%%%%%%%%%%%%%%%%%%%%%%%%%%%%%%%%%%%%%%%%%%%%%%%%%%%%%%%%%%%%%%%%%%%%%%%%


\end{document}
