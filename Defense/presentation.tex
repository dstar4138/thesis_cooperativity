\documentclass{beamer}
\usepackage{etex}
\usepackage{url}
\usepackage{semantic}
\usepackage{pgfpages}
\usepackage{multicol}\columnseprule 0.4pt
\usepackage{subfigure}
\usepackage{tikz,tikz-qtree}
\usepackage{verbatim,fancyvrb}
\usepackage{mathtools,mathrsfs,amsmath,amsthm,amsfonts}

%% Symbolic, relative link to thesis graphics.
\graphicspath{ {./pics/}{./extra/} }

%% Optional: Turn on Notes Pages:
\setbeameroption{show notes on second screen}

%% Beamer Theme Options
\usetheme{Rochester} % Because RIT.
\usecolortheme[named=black]{structure} % Default

%% Document Information:
\title{Process Cooperativity as a Feedback Metric\\
        in Concurrent Message-Passing Languages}
\author{Alexander Dean}
\institute{
    Rochester Institute of Technology\\
    Golisano College of Computer and Information Sciences
}
\date{August 12, 2014} % Scheduled Defense Date
\subject{Thesis Defense}

%% Document Table-Of-Contents, and pacing slides.
\AtBeginSection[]
{
    \begin{frame}
        \frametitle{\insertsection}
        \begin{multicols*}{2}
            \tableofcontents[sectionstyle=show/hide,
                             subsectionstyle=show/show/hide]
        \end{multicols*}
        \note{Introduce the new section:\hfill
              \tableofcontents[sectionstyle=show/hide,
                               subsectionstyle=show/show/hide]}
    \end{frame}
}

%% Slides have titles based on their section and subsection.
\newenvironment{slide}
    {\begin{frame}[environment=slide,fragile]
            \frametitle{\insertsection: \insertsubsection}
    }
    {\end{frame}}

%% Itemized Notes:
\newcommand{\inote}[1]{\note{\begin{itemize} #1 \end{itemize}}}

%% Abbreviations:
\newcommand{\etc} {\emph{etc.\/}}
\newcommand{\etal}{\emph{et~al.\/}}
\newcommand{\eg}  {\emph{e.g.\/}}
\newcommand{\ie}  {\emph{i.e.\/}}

\begin{document}

% TITLE PAGE:
\frame{\titlepage
    \inote{
        \item Thank Fluet, Heliotis, and Raj.
        \item Dedicate to parents, who are unable to be present.
    }
}

% ToC PAGE:
\begin{frame}
    \frametitle{Process Cooperativity as a Feedback Metric}
    \framesubtitle{in Concurrent Message-Passing Languages}
    \begin{multicols*}{2}
        \tableofcontents
    \end{multicols*}
    \inote{
        \item Mouthful of a title, so I'll break it up:
        \begin{enumerate}{\scriptsize
            \item Runtime scheduling, to give some grounding in the area of study.
            \item Cooperativity, what it is and motivation to use it.
            \item Message Passing, because, as it turns out, it's a nice 
                abstraction for our purpose of capturing cooperativity.
            }
        \end{enumerate}
        \item The core of the work revolves around the toolkit I built.
            \begin{itemize}{\scriptsize
                \item A language/compiler/runtime/testing-framework
                \item But also a {\sl Simulator} which has a plug-and-play 
                    scheduler API. It let me test schedulers on a common test 
                    bed.
                }
            \end{itemize}
        \item Next, go over the list of schedulers \& feedback mechanisms.
        \item Results, Conclusions, \& Future Work.
    }
\end{frame}

% INCLUDE EACH SECTION:
\section{Background}

%%%%%%%%%%%%%%%%%%%%%%%%%%%%%%%%%%%%%%%%%%%%%%%%%%%%%%%%%%%%%%%%%%%%%%%%%%%%%%
\subsection{Cooperativity}

\begin{slide}
    What is Process Cooperativity?
    \begin{figure}
        \subfigure{
            \includegraphics[scale=0.4]{ChugMachine.pdf}
        }
        \hspace{10mm}
        \subfigure{
            \includegraphics[scale=0.5]{SingleCluster.pdf}
        }
    \end{figure}

    \inote{
        \item Where white represents a channel and black represents a process.
        \item Channel = Source of synchronization (\ie~locks, abstractions, \etc)
        \item[] ~
        \item Left: Cloud of processes with no interaction.
        \item Right: Some sort of batch job? A bit of shared state amongst all processes
                these processes could all really be parallel or be competing, but in 
                either case they cooperate.
        \item DEGREE OF COOPERATIVITY: 
            \begin{itemize}
                \item {\em of process:} flux of interaction with inter-proc comm method.
                \item {\em of system:} rate of interaction between Ps and Cs as both flux in size.
            \end{itemize}
    }
\end{slide}


\begin{slide}
    What does Cooperativity give us?
    \begin{figure} 
    \centering
        \includegraphics[scale=0.4]{RingVCluster.pdf}
        \label{fig:RingVCluster}
    \end{figure}

    \note{
        Whats the difference in cooperativity in the left and right
        set of processes?         \begin{itemize}
            \item Left: A Ring, the level of parallelism is nearly nil. Each
                process is cooperating yes, but the granularity of the 
                application is very fine.

            \item Right: A Star, the level of parallelism is nearly full. Each
                process is cooperating, and is {\bf not reliant on more than one} 
                other.
        \end{itemize}
        Overall, what can be gained by looking at cooperativity in terms of
        understanding the applications behaviour? Knowing the granularity of
        parallelism.\\
        ~\\
        \hfill
        Next: what is a minimal inter-proc comm method?
    }
\end{slide}


%%%%%%%%%%%%%%%%%%%%%%%%%%%%%%%%%%%%%%%%%%%%%%%%%%%%%%%%%%%%%%%%%%%%%%%%%%%%%%
\subsection{Message Passing}

\begin{slide}
    We use a Symmetric, Synchronous, Message-Passing Primitive: 
            \begin{center} {\tt\large swap} \end{center}
    \begin{itemize}
        \item Purely captures cooperation of processes by synchronizing on
                the shared channel.
        \item Ultimately can be extended to take into account:
            \begin{itemize}
                \item Directionality
                \item Asynchrony
            \end{itemize}
    \end{itemize}
    \inote{
       \item Async: while user code doesn't block, there is blocking
             in terms of the channel implementation. This is not
             indicated (in most cases) to the scheduler.
       \item Sync: The issue of process cooperation has been elevated
             to the process level for the scheduler to directly involve
             itself.
       \item
        Ultimately there is nothing stopping us from choosing the other
        types of message passing, but it would conflate the issue of
        process cooperativity if all we are after is whether two processes
        are cooperating on some task.\\
        ~\\
        \hfill
        Next: How would we use coop as a feedback metric?\\ 
        \hfill
        What does that mean?
    }
\end{slide}


%%%%%%%%%%%%%%%%%%%%%%%%%%%%%%%%%%%%%%%%%%%%%%%%%%%%%%%%%%%%%%%%%%%%%%%%%%%%%%
\subsection{Runtime Scheduling}

\begin{slide}
    \begin{itemize}
        \item Schedulers can be defined in a discrete manner:
            \begin{enumerate}
                \item {\em Choose} a process from set
                \item {\em Reduce} it
                \item {\em Update} private scheduler state
            \end{enumerate}
        \item Statistics can be gathered at every step about process:
            \begin{itemize}
                \item Number of channel partners,
                \item Timestamp of last run,
                \item Number of reductions, \etc
            \end{itemize}
        \item {\sl What statistics are neccessary for recognizing cooperativity?}
    \end{itemize}

    \inote{
        \item Choosing from set: Top Always (batching), Queue (Round-Robin)
        \item Reductions Return some indication: yield/blocked, unblocked, completed, nothing
        \item Update state based on historical data.
            \begin{itemize}
                \item Timestamp of last run = Separate between batching/round-robin
                \item Cooperativity metrics: yields, partners, thus longevity and granularity 
            \end{itemize}
        \item gets back to why we chose swap:
            \begin{itemize}
                \item (if asymmetrical, we would have a harder time with 'partner')
                \item (if async, we would have a harder time with 'yield' and noticing longevity/progress)
            \end{itemize}
    }
\end{slide}


\section{ErLam}

\subsection{The Language}

\begin{slide}
    \begin{figure}
    \centering
        {\footnotesize
            %%
%% ErLam BNF Style Grammar.
%%
\begin{BVerbatim}[commandchars=\\\{\}]
<Expression> ::= <Variable> 
              |  <Integer>
              |  `\textbf{newchan}'
              |  `\textbf{(}' <Expression> `\textbf{)}'
              |  <Expression> <Expression>
              |  `\textbf{if}' <Expression> <Expression> <Expression>
              |  `\textbf{swap}' <Channel> <Expression>
              |  `\textbf{spawn}' <Expression>
              |  `\textbf{fun}' <Variable> `\textbf{.}' <Expression>
\end{BVerbatim}

        }
    \label{fig:grammer}
    \end{figure}

    \inote{
        \item Extremely simple on purpose (5 keywords).
        \item Issue now began to be how to build up test primitives
        \item Made a library which allowed for built ins.
    }
\end{slide}

\begin{slide}
    \begin{figure}
    \centering
    \begin{BVerbatim}
elib
    // ...
    omega = (fun x.(x x));
    // ...
    add = _erl[2]{ fun(X) when is_integer(X) ->
                        fun(Y) when is_integer(Y) ->
                            X+Y
                        end
                    end
                 };
    // ...
bile
    \end{BVerbatim}
    \end{figure}

    \inote{
        \item Theres options for built-ins as well as macros.
    }
\end{slide}

\begin{SaveVerbatim}[commandchars=\\\{\}]{FibCode}
// pfib.els -
fun N.
    (omega \textbf{fun} f,m.(
        \textbf{if} (leq m 1) 
           m
           (merge \textbf{fun} _.(f f (sub m 1))
                  \textbf{fun} _.(f f (sub m 2))
                  add)) \textit{N})
\end{SaveVerbatim}

\begin{slide}
\framesubtitle{Example Application: Parallel Fibonacci}
    \begin{figure}
    \centering
    {\small
        \BUseVerbatim{FibCode}
    } 
    \end{figure}
    \begin{itemize}
        \item[] {\tt \$ els pfib.els}\hspace{8.65mm}(Compile the script)
        \item[] {\tt \$ ./pfib.ex -r 10}~~(Finds the 10th Fibonacci number)
    \end{itemize}
    
    \inote{
        \item R option is to run program applied to 10.
        \item To add more to this presentation than just reading paper
            I want to also give more detail as to system usage.
        \item Explain this Common Usage Pattern.
    }
\end{slide}

%%%%%%%%%%%%%%%%%%%%%%%%%%%%%%%%%%%%%%%%%%%%%%%%%%%%%%%%%%%%%%%%%%%%%%%%%%%%%%
\subsection{Channel Implementations}

\begin{slide}
    \begin{figure}
        \makebox[\textwidth][c]{
            \subfigure[t][Process Blocking Swap]{
                \includegraphics[width=0.5\textwidth]{BlockingSwap.pdf}
                \label{fig:blockchan-example}
            }  \subfigure[t][Process Absorption Swap]{
                \includegraphics[width=0.5\textwidth]{AbsorbSwap2.pdf}
                \label{fig:absorbchan-example}
            }
        }
    \end{figure}
    \inote{
        \item Blocking: Maintains state of current and previous swap value until swap is completed.
        \item Absorption: Stores process which initializes the swap and returns it along with completion.
        \item Mention expected effects on scheduler.
        \item Blocking is default, passing '-a' to els 
    }
\end{slide}


%%%%%%%%%%%%%%%%%%%%%%%%%%%%%%%%%%%%%%%%%%%%%%%%%%%%%%%%%%%%%%%%%%%%%%%%%%%%%%
\subsection{Simulation \& Visualization}

\begin{slide}


    \inote{
        \item Need to talk about Chart types and chart generation.
    }
\end{slide}





















\section{Scheduler Implementations}

\subsection{Example Schedulers}

\begin{slide}
    \begin{itemize}
        \item (MTRRGQ) - Round-Robin with Single Global Queue
            \begin{itemize}
                \item All LPUs share a Process Queue.
            \end{itemize}

        \item (MTRRWS-SQ) - Round-Robin with Work-Stealing via Direct Access
            \begin{itemize}
                \item All LPUs have their own Process Queue.
                \item LPUs can steal processes by grabbing them off 
                    the end of another LPU's queue.
            \end{itemize} 
    \end{itemize}
    \note{
        Two of the basic schedulers built where:
        \begin{enumerate}
        \item RR w/ Global Queue: all synchronization around a single shared queue
        \item RR w/ Work-Stealing: each scheduler gets their own queue but,
                they now need to steal work from others.
            \begin{itemize}
                \item Implemented multiple types of work stealing, but we'll
                    limit talk to one type:
                \item Stealing directly from another LPUs by accessing the end
                    of their process queue. 
            \end{itemize}
        \end{enumerate}
    }
\end{slide}

\subsection{Feedback Mechanisms}

\begin{slide}
    Three types of mechanics:
    \begin{itemize}
        \item Longevity-Based Batching
        \item Channel Pinning
        \item Bipartite-Graph Aided Sorting
    \end{itemize}

    \inote{
        \item Instead of a single cooperativity-conscious scheduler,
            we implemented three mechanics which take cooperativity into
            account on top of the basic schedulers.
   }    
\end{slide}

\begin{slide}
    \framesubtitle{Longevity-Based Batching}

    \begin{itemize}
        \item Choose via Round-Robin 
            \begin{itemize}
                \item from batch rather than queue
                \item keeps track of number of rounds (batch size)
            \end{itemize}

        \item Work-Steal whole batches

        \item Spawn to batch unless: $|b_i| \geq B$
            \begin{itemize}
                \item Make singleton with new process.
                \item Push parent and child into new batch.
            \end{itemize}

        \item[GOAL:]<2-> Can batching based on longevity account for fine/coarse
            parallelism in application? 

     \end{itemize}    
    
    \inote{
        \item Batching processes based on longevity.
            \begin{itemize}
                \item Based on occam-$\Pi$.
                \item if a process communicates frequently then
                    it will be batched (absorption), singleton if 
                    very computation-bound.
            \end{itemize}
        \item We are normal RR but with one extra layer.
        \item If batch is too big during spawns we can:
            \begin{itemize}
                \item Make singleton, best if child is needed to 
                    start work right away. Map-Reduce.
                \item Make push-back, parent can get another chance
                    to spawn more children sooner.
            \end{itemize}
    }
\end{slide}

\begin{slide}
    \framesubtitle{Channel-Pinning}

    \begin{itemize}
        \item Upon call to $newchan$, pin to LPU based on spread algorithm:
            \begin{itemize}
                \item $same$ - LPU $newchan$ is called is where it is pinned.
                \item $even$ - Cycle through LPUs and pin based on that.
                \item \ldots
            \end{itemize}

        \item Work-steal based on channel that's been pinned to you.

        \item[GOAL:]<2-> Can an $even$-like spread increase early saturation?
    \end{itemize}

    \inote{
        \item Pin channels to LPUs.
            \begin{itemize}
                \item Pinning a channel means to set a process affinity to a 
                      LPU based on the channels it uses.
                \item Work-Stealing works like Go-Fish.
            \end{itemize}
    }
\end{slide}


\begin{slide}
    \framesubtitle{Bipartite-Graph Aided Sorting}
   
    \begin{multicols*}{2} 
        \begin{itemize}
            \item Based on Round-Robin \& Work-stealing
            \item Keep track of events which may effect cooperativity:
                \begin{itemize}
                    \item Spawning
                    \item Blocking/Unblocking
                    \item Steals
                \end{itemize}
            \item If number of events over some threshold, re-sort. 
        \end{itemize} 
    
        \begin{figure}
            \centering
            \includegraphics[scale=0.5]{BipartiteGraph.pdf}
            \vspace{5mm}
        \end{figure}
    \end{multicols*}

    \begin{itemize}
        \item[GOAL:]<2-> Are alternate channel implementations worth exploration?
    \end{itemize}

    \inote{
        \item Keep a list of all communications as a graph between set of processes and channels.
    }
\end{slide}


\section{Results}

\begin{slide}
    Longevity-Based Batching Scheduler:
    \begin{itemize}
    \item Spawn Mechanism 
    \item[] ~~~~{\it (Where does a new process go if batch is too big?)}
    \end{itemize}
    
    Channel Pinning Scheduler:
    \begin{itemize}
    \item Channel Spread 
    \item[] ~~~~{\it (How to spread the channels across processors?)}
    \end{itemize}

    Bipartite-Graph Aided Sorting Scheduler:
    \begin{itemize}
    \item Channel Implementation 
    \item[] ~~~~{\it (Does blocking help to take advantage of sorting?)}
    \end{itemize}

    \inote{
        \item LBB: Can also look at how the batch size effects different types
            of behaviours.
        \item SS: Could also look at a stealing mechanism.
    }
\end{slide}

\begin{slide}
\framesubtitle{Longevity-Based Batching Scheduler}

        \begin{table}[htp!]
            \centering
            \begin{tabular}{@{}cccc}
                & \multicolumn{3}{c}{$PRing_N$} \\ \cline{2-4}
            & $N=P=8$ & $N=B=10$ & $N=2*B=20$     \\ \cline{2-4} 
                \multicolumn{1}{c|}{\rotatebox{90}{\rlap{\textbf{Reduc. Density}}}} & 
            \multicolumn{1}{c|}{\includegraphics[scale=0.15]{tests/pring/longbatcher/8/pg_0004.pdf}} & 
            \multicolumn{1}{c|}{\includegraphics[scale=0.15]{tests/pring/longbatcher/10/pg_0004.pdf}} & 
            \multicolumn{1}{c|}{\includegraphics[scale=0.15]{tests/pring/longbatcher/20/pg_0004.pdf}} \\ 
            \cline{2-4} 
        \end{tabular}
        \caption{Comparison of different sized $PRing_N$ on the Longevity 
                 Batching Scheduler with batch size $B=10$.}
            \label{tab:pring-longbatcher-testing}
        \end{table}

    \inote{
        \item Before talking about spawn mechanism, pointing out the
              primary goal and effect of batching to catch the granularity
              of the application.
    }
\end{slide}
\begin{slide}
\framesubtitle{Longevity-Based Batching Scheduler}
    \inote{
        \item TODO: Get simple $ClusterComm_{(N,1)}$ or the PTree results to contrast with PRING 
        \item This brings it back to an issue of behaviour recognition.
    }
\end{slide}


\begin{slide}
\framesubtitle{Channel Pinning Scheduler}
    \begin{table}
    \centering
    \begin{tabular}{@{}ccc}
    & \multicolumn{2}{c}{$Interactivity_{(20,0)}$} \\ \cline{2-3} 
    & \multicolumn{1}{c}{$MTRRWS$-$SQ$}       & \multicolumn{1}{c}{Channel Pinning} \\ \cline{2-3} 
 
    \multicolumn{1}{c|}{\rotatebox{90}{\rlap{~~Queue Length}}} &
    \multicolumn{1}{c}{\includegraphics[scale=0.15]{tests/interactivity/20/wssq/ca/pg_0003.pdf}} & 
    \multicolumn{1}{c|}{\includegraphics[scale=0.15]{tests/interactivity/20/cp/ca/pg_0003.pdf}} \\

    \multicolumn{1}{c|}{\rotatebox{90}{\rlap{Reduc. Density}}} &
    \multicolumn{1}{c}{\includegraphics[scale=0.15]{tests/interactivity/20/wssq/ca/pg_0004.pdf}} & 
    \multicolumn{1}{c|}{\includegraphics[scale=0.15]{tests/interactivity/20/cp/ca/pg_0004.pdf}} \\ \cline{2-3}
    \end{tabular}
    \label{tab:cp-compare-rand-uniform-ca}
    \end{table}

    \inote{
        \item Comparison of Uniform synchronization for $MTRRWS$-$SQ$ 
                and the Channel Pinning Scheduler on Absorption Channels.
        \item This used the $even$ spread type.
        \item Note the speed at which it saturates all cores.
        \item Despite Naive WS, we still have decent spread.
    }
\end{slide}

\begin{slide}
\framesubtitle{Bipartite-Aided Graph Sorting Scheduler}

    \begin{table}
    \centering
    \begin{tabular}{@{}ccc}
        & \multicolumn{2}{c}{Parallel Fibonacci} \\ \cline{2-3}
        & $MTRRWS$-$SQ$ & Sorting Scheduler  \\ \cline{2-3} 
        \multicolumn{1}{c|}{~}  & 
        \multicolumn{1}{c|}{\includegraphics[scale=0.27]{tests/pfib/wssq/pg_0001.pdf}} & 
        \multicolumn{1}{c|}{\includegraphics[scale=0.27]{tests/pfib/ss/pg_0001.pdf}} \\ \cline{2-3}
    \end{tabular}
    \label{tab:ss-compare-fib}
    \end{table}

    \inote{
        \item Channel State comparison of Parallel Fibonacci executed on 
                $MTRRWS$-$SQ$ and the Bipartite-Graph Aided Sorting Scheduler. 
        \item Note the large reduction in number of ticks.
    }
\end{slide}


\section{Conclusions \& Future Work}

\subsection{ErLam Toolkit}

\begin{slide}
    \begin{itemize}
        \item Test Primitives proved to be a good abstraction for 
            Process behaviours.
        \item Log generation, lots of overhead, but good observations.
    \end{itemize}

    \inote{
        \item Overall pleased with simulator and achieved its goal.
        \item Future Work: 
            \begin{itemize}
                \item Generate more test primitives. 
                \item Clean up log generation (reduce overhead).
                \item Process evaluation uses alpha-reduction 
                      (can be sped up substantially).
                \item Make schedulers more adjustable (different
                        spawn/yields/etc).
                \item More Channel implementations
            \end{itemize}
    }
\end{slide}

\subsection{Cooperativity as a Metric}

\begin{slide}

    \begin{itemize}
        \item Possible to recognize and benefit from.
        \item More to explore:
            \begin{itemize}
                \item Alternate Message-Passing Types (Asymmetric?)
                \item ....
            \end{itemize}
    
    \end{itemize}

\end{slide}

\subsection{Cooperative Schedulers}

\begin{slide}
    \begin{itemize}
        \item {\bf Longevity Batching:} -  
        \item {\bf Channel Pinning:} -  
        \item {\bf Queue Sorting:} Would benefit from merging with
            Channel Pinning scheduler to increase likelihood that
            sorting puts channel partners close. 
    \end{itemize}

    \inote{
        \item Merging Schedulers
    }
\end{slide}



% Questions and Thank you!
\section*{Questions}
\begin{frame}
    \begin{center}
        {\Huge Questions/Comments?}
    \end{center}

    \uncover<2->{
        \begin{center}
            Thank You!
        \end{center}
    }
\end{frame}

% Bibliography and Links Sections:
%\input{sections/999bibliography}
\begin{frame}
    \frametitle{Links}
    \begin{itemize}
        \item \url{https://github.com/dstar4138/erlam}
        \item \url{https://github.com/dstar4138/thesis_cooperativity}
        \item \url{http://dstar4138.com}
    \end{itemize}
\end{frame}


\end{document}
