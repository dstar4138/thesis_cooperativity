%%%%%%%%%%%%%%%%%%%%%%%%%%%%%%%%%%%%%%%%%%%%%%%%%%%%%%%%%%%%%%%%%%%%%%
%%  disstemplate.tex, to be compiled with latex.             %
%%  08 April 2002   Version 4                    %
%%%%%%%%%%%%%%%%%%%%%%%%%%%%%%%%%%%%%%%%%%%%%%%%%%%%%%%%%%%%%%%%%%%%%%
%%                                   %
%%  Writing a Doctoral Dissertation with LaTeX at            %
%%  the University of Texas at Austin                %
%%                                   %
%%  (Modify this ``template'' for your own dissertation.)        %
%%                                   %
%%%%%%%%%%%%%%%%%%%%%%%%%%%%%%%%%%%%%%%%%%%%%%%%%%%%%%%%%%%%%%%%%%%%%%


\documentclass[12pt]{report}    % The documentclass must be ``report''.

\usepackage{utdiss2}        % Dissertation package style file.


%%%%%%%%%%%%%%%%%%%%%%%%%%%%%%%%%%%%%%%%%%%%%%%%%%%%%%%%%%%%%%%%%%%%%%
% Optional packages used for this sample dissertation. If you don't  %
% need a capability in your dissertation, feel free to comment out   %
% the package usage command.                         %
%%%%%%%%%%%%%%%%%%%%%%%%%%%%%%%%%%%%%%%%%%%%%%%%%%%%%%%%%%%%%%%%%%%%%%

\usepackage{amsmath,amsthm,amsfonts,amscd}
                % Some packages to write mathematics.
\usepackage{eucal}      % Euler fonts
\usepackage{verbatim}       % Allows quoting source with commands.
\usepackage{makeidx}        % Package to make an index.
\usepackage{psfig}          % Allows inclusion of eps files.
\usepackage{epsfig}             % Allows inclusion of eps files.
\usepackage{citesort}           %
\usepackage{url}        % Allows good typesetting of web URLs.
%\usepackage{draftcopy}     % Uncomment this line to have the
                % word, "DRAFT," as a background
                % "watermark" on all of the pages of
                % of your draft versions. When ready
                % to generate your final copy, re-comment
                % it out with a percent sign to remove
                % the word draft before you re-run
                % Makediss for the last time.

\author{Amit Pillay}    % Required

\address{1814 Crittenden Road Apt 6 \\ Rochester, New York 14623}  % Required

\title{Intelligent Combination of Structural Analysis Algorithms:\\
 Application to Mathematical Expression Recognition}
                                                    % Required

%%%%%%%%%%%%%%%%%%%%%%%%%%%%%%%%%%%%%%%%%%%%%%%%%%%%%%%%%%%%%%%%%%%%%%
% NOTICE: The total number of supervisors and other members %%%%%%%%%%
%%%%%%%%%%%%%%% MUST be seven (7) or less! If you put in more, %%%%%%%
%%%%%%%%%%%%%%% they are put on the page after the Committee %%%%%%%%%
%%%%%%%%%%%%%%% Certification of Approved Version page. %%%%%%%%%%%%%%
%%%%%%%%%%%%%%%%%%%%%%%%%%%%%%%%%%%%%%%%%%%%%%%%%%%%%%%%%%%%%%%%%%%%%%

%%%%%%%%%%%%%%%%%%%%%%%%%%%%%%%%%%%%%%%%%%%%%%%%%%%%%%%%%%%%%%%%%%%%%%
%
% Enter names of the supervisor and co-supervisor(s), if any,
% of your dissertation committee. Put one name per line with
% the name in square brackets. The name on the last line, however,
% must be in curly braces.
%
% If you have only one supervisor, the entry below will read:
%
%   \supervisor
%       {Supervisor's Name}
%
% NOTE: Maximum three supervisors. Minimum one supervisor.
% NOTE: The Office of Graduate Studies will accept only two supervisors!
%
%
\supervisor
    {Dr. Richard Zanibbi}

%%%%%%%%%%%%%%%%%%%%%%%%%%%%%%%%%%%%%%%%%%%%%%%%%%%%%%%%%%%%%%%%%%%%%%
%
% Enter names of the other (non-supervisor) members(s) of your
% dissertation committee. Put one name per line with the name
% in square brackets. The name on the last line, however, must
% be in curly braces.
%
% NOTE: Maximum six other members. Minimum zero other members.
% NOTE: The Office of Graduate Studies may restrict you to a total
%   of six committee members.
%
%
\committeemembers
%    [Erwin Schr\"odinger]
    [ABC, Reader]
    {XYZ, Observer}
%    {Arthur Schawlow}

%%%%%%%%%%%%%%%%%%%%%%%%%%%%%%%%%%%%%%%%%%%%%%%%%%%%%%%%%%%%%%%%%%%%%%

\previousdegrees{B.E.}
     % The abbreviated form of your previous degree(s).
     % E.g., \previousdegrees{B.S., MBA}.
     %
     % The default value is `B.S., M.S.'

%\graduationmonth{...}
     % Graduation month, either May, August, or December, in the form
     % as `\graduationmonth{May}'. Do not abbreviate.
     %
     % The default value (either May, August, or December) is guessed
     % according to the time of running LaTeX.

%\graduationyear{...}
     % Graduation year, in the form as `\graduationyear{2001}'.
     % Use a 4 digit (not a 2 digit) number.
     %
     % The default value is guessed according to the time of
     % running LaTeX.

%\typist{...}
     % The name(s) of typist(s), put `the author' if you do it yourself.
     % E.g., `\typist{Maryann Hersey and the author}'.
     %
     % The default value is `the author'.


%%%%%%%%%%%%%%%%%%%%%%%%%%%%%%%%%%%%%%%%%%%%%%%%%%%%%%%%%%%%%%%%%%%%%%
% Commands for master's theses and reports.              %
%%%%%%%%%%%%%%%%%%%%%%%%%%%%%%%%%%%%%%%%%%%%%%%%%%%%%%%%%%%%%%%%%%%%%%
%
% If the degree you're seeking is NOT Doctor of Philosophy, uncomment
% (remove the % in front of) the following two command lines (the ones
% that have the \ as their second character).
%
\degree{Master of Science} \degreeabbr{M.S.}

% Uncomment the line below that corresponds to the type of master's
% document you are writing.
%
%\masterreport
\masterthesis


%%%%%%%%%%%%%%%%%%%%%%%%%%%%%%%%%%%%%%%%%%%%%%%%%%%%%%%%%%%%%%%%%%%%%%
% Some optional commands to change the document's defaults.      %
%%%%%%%%%%%%%%%%%%%%%%%%%%%%%%%%%%%%%%%%%%%%%%%%%%%%%%%%%%%%%%%%%%%%%%
%
%\singlespacing
%\oneandonehalfspacing

%\singlespacequote
\oneandonehalfspacequote

\topmargin 0.125in  % Adjust this value if the PostScript file output
            % of your dissertation has incorrect top and
            % bottom margins. Print a copy of at least one
            % full page of your dissertation (not the first
            % page of a chapter) and measure the top and
            % bottom margins with a ruler. You must have
            % a top margin of 1.5" and a bottom margin of
            % at least 1.25". The page numbers must be at
            % least 1.00" from the bottom of the page.
            % If the margins are not correct, adjust this
            % value accordingly and re-compile and print again.
            %
            % The default value is 0.125"

        % If you want to adjust other margins, they are in the
        % utdiss2-nn.sty file near the top. If you are using
        % the shell script Makediss on a Unix/Linux system, make
        % your changes in the utdiss2-nn.sty file instead of
        % utdiss2.sty because Makediss will overwrite any changes
        % made to utdiss2.sty.

%%%%%%%%%%%%%%%%%%%%%%%%%%%%%%%%%%%%%%%%%%%%%%%%%%%%%%%%%%%%%%%%%%%%%%
% Some optional commands to be tested.                   %
%%%%%%%%%%%%%%%%%%%%%%%%%%%%%%%%%%%%%%%%%%%%%%%%%%%%%%%%%%%%%%%%%%%%%%

% If there are 10 or more sections, 10 or more subsections for a section,
% etc., you need to make an adjustment to the Table of Contents with the
% command \longtocentry.
%
%\longtocentry



%%%%%%%%%%%%%%%%%%%%%%%%%%%%%%%%%%%%%%%%%%%%%%%%%%%%%%%%%%%%%%%%%%%%%%
%   Some math support.                       %
%%%%%%%%%%%%%%%%%%%%%%%%%%%%%%%%%%%%%%%%%%%%%%%%%%%%%%%%%%%%%%%%%%%%%%
%
%   Theorem environments (these need the amsthm package)
%
%% \theoremstyle{plain} %% This is the default

\newtheorem{thm}{Theorem}[section]
\newtheorem{cor}[thm]{Corollary}
\newtheorem{lem}[thm]{Lemma}
\newtheorem{prop}[thm]{Proposition}
\newtheorem{ax}{Axiom}

\theoremstyle{definition}
\newtheorem{defn}{Definition}[section]

\theoremstyle{remark}
\newtheorem{rem}{Remark}[section]
\newtheorem*{notation}{Notation}

%\numberwithin{equation}{section}


%%%%%%%%%%%%%%%%%%%%%%%%%%%%%%%%%%%%%%%%%%%%%%%%%%%%%%%%%%%%%%%%%%%%%%
%   Macros.                              %
%%%%%%%%%%%%%%%%%%%%%%%%%%%%%%%%%%%%%%%%%%%%%%%%%%%%%%%%%%%%%%%%%%%%%%
%
%   Here some macros that are needed in this document:


\newcommand{\latexe}{{\LaTeX\kern.125em2%
                      \lower.5ex\hbox{$\varepsilon$}}}

\newcommand{\amslatex}{\AmS-\LaTeX{}}

\chardef\bslash=`\\ % \bslash makes a backslash (in tt fonts)
            %   p. 424, TeXbook

\newcommand{\cn}[1]{\texttt{\bslash #1}}

\makeatletter       % Starts section where @ is considered a letter
            % and thus may be used in commands.
\def\square{\RIfM@\bgroup\else$\bgroup\aftergroup$\fi
  \vcenter{\hrule\hbox{\vrule\@height.6em\kern.6em\vrule}%
                                              \hrule}\egroup}
\makeatother        % Ends sections where @ is considered a letter.
            % Now @ cannot be used in commands.

\makeindex    % Make the index

%%%%%%%%%%%%%%%%%%%%%%%%%%%%%%%%%%%%%%%%%%%%%%%%%%%%%%%%%%%%%%%%%%%%%%
%       The document starts here.                %
%%%%%%%%%%%%%%%%%%%%%%%%%%%%%%%%%%%%%%%%%%%%%%%%%%%%%%%%%%%%%%%%%%%%%%

\begin{document}

%\copyrightpage          % Produces the copyright page.


%
% NOTE: In a doctoral dissertation, the Committee Certification page
%       (with signatures) is BEFORE the Title page.
%   In a masters thesis or report, the Signature page
%       (with signatures) is AFTER the Title page.
%
%   If you are writing a masters thesis or report, you MUST REVERSE
%   the order of the \commcertpage and \titlepage commands below.
%
\commcertpage           % Produces the Committee Certification
            %   of Approved Version page (doctoral)
            %   or Signature page (masters).
            %       20 Mar 2002 cwm

\titlepage              % Produces the title page.



%%%%%%%%%%%%%%%%%%%%%%%%%%%%%%%%%%%%%%%%%%%%%%%%%%%%%%%%%%%%%%%%%%%%%%
% Dedication and/or epigraph are optional, but must occur here.      %
%%%%%%%%%%%%%%%%%%%%%%%%%%%%%%%%%%%%%%%%%%%%%%%%%%%%%%%%%%%%%%%%%%%%%%
%
%\begin{dedication}
%\index{Dedication@\emph{Dedication}}%
%Dedicated to my wife Shirley.
%\end{dedication}


\begin{acknowledgments}     % Optional
\index{Acknowledgments@\emph{Acknowledgments}}%
I wish to thank the multitudes of people who helped me. Time would
fail me to tell of \ldots
\end{acknowledgments}


% The abstract is required. Note the use of ``utabstract'' instead of
% ``abstract''! This was necessary to fix a page numbering problem.
% The abstract heading is generated automatically.
% Do NOT use \begin{abstract} ... \end{abstract}.
%
\utabstract
\index{Abstract}%
\indent
This document has the form of a ``fake'' doctoral
dissertation in order to provide an example of such, but it is
actually a copy of Miguel Lerma's documentation for the Mathematics
Department Computer Seminar of 25 March 1998 updated in July 2001
and following by Craig McCluskey to meet the March 2001 requirements
of the Graduate School.

This document and its source file show to write a Doctoral Dissertation using
\LaTeX{} and the utdiss2 package.



\tableofcontents   % Table of Contents will be automatically
                   % generated and placed here.

\listoftables      % List of Tables and List of Figures will be placed
\listoffigures     % here, if applicable.



%%%%%%%%%%%%%%%%%%%%%%%%%%%%%%%%%%%%%%%%%%%%%%%%%%%%%%%%%%%%%%%%%%%%%%
% Actual text starts here.                       %
%%%%%%%%%%%%%%%%%%%%%%%%%%%%%%%%%%%%%%%%%%%%%%%%%%%%%%%%%%%%%%%%%%%%%%
%
% Including external files for each chapter makes this document simpler,
% makes each chapter simpler, and allows for generating test documents
% with as few as zero chapters (by commenting out the include statements).
% This allows quicker processing by the Makediss command file in case you
% are not working on a specific, long and slow to compile chapter. You
% can even change the chapter order by merely interchanging the order
% of the include statements (something I found helpful in my own
% dissertation).
%
\chapter{Introduction}
\index{Introduction@\emph{Introduction}}%

\begin{comment}
 This document deals with how to write a doctoral dissertation
using \LaTeX{}, and how to use the \texttt{utdiss2} package.
\index{utdiss2 package@{\texttt{utdiss2} package}}%

The latest version of this document/package can be obtained from
\url{http://www.ph.utexas.edu/~laser/craigs_stuff/LaTeX/}.\footnote{I
will be transferring this page to the Office of Graduate
Studies when I graduate. The new URL isn't defined yet, but I will
place a ``redirect'' at this URL to send your browser to the correct
location when the transition occurs.}
If your installation of LaTeX is missing any style files used in this
document (most likely with a \cn{usepackage\{package-name.sty\}}
command at the beginning of disstemplate.tex), take a look at the link
on this page to ``Frequently Requested Style Files'' or on the
Comprehensive TeX Archive Network, \url{http://www.ctan.org}.

In case of any confict between the requirements of the Office of Graduate
Studies and what this document says to do, the requirements of the Office
of Graduate Studies prevail.

\section{History of This Package}
\index{History of This Package@\emph{History of This Package}}%

In 1991 the \texttt{utdiss} package was written by Young U. Ryu
\index{Young U. Ryu}%
in order to be used in the preamble of \LaTeX{} doctoral dissertation
files at the University of Texas at Austin.
\index{University of Texas at Austin}%
Since then some changes have occured, the most important one
being the introduction of a new version of \LaTeX{}
\index{LaTeX@{\LaTeX{}}}%
called \LaTeXe{}.
\index{LaTeX2e@{\LaTeXe{}}}%

In order to partially adapt the utdiss package to this new version
of \LaTeX{}, Miguel Lerma introduced a few modifications in it,
and his document, \textit{How to Write a Doctoral Dissertation
with \LaTeX{}}, served as a test for it. His new package was
called \texttt{utdiss1}.
\index{utdiss1@\texttt{utdiss1}}%

With the significant changes in style introduced by the Graduate
School in the Spring of 2001, as well as  my need to write a
dissertation myself, I extended Miguel Lerma's package to meet
these new requirements. As in Miguel Lerma's case, this document
serves as a test for it, but it is, in addition, intended as a
template for others to use in writing their own dissertations.
The new package is called \texttt{utdiss2}.
\index{utdiss2@\texttt{utdiss2}}%

\section{Revised Philosophy for This Package}
\index{Revised Philosophy for This Package@\emph{Revised Philosophy
    for This Package}}%

Since the source file of this document is intended to be used by students
writing their own dissertations, this document does not display all of the
comments regarding usage of previous versions. It has, instead, transferred
these comments to their respective places in the source file so someone
editing their own copy of the source file to produce their own dissertation
will see the comments where they are needed. It may be helpful to print out
a copy of the source file along with the PostScript version of the document
so the two can be studied side-by-side.

\textbf{Note:} In spite of the effort to accommodate the package to
the requirements of the University, it is not possible to guarantee
that it will always work, and the author of the dissertation remains
responsible for checking that such requirements are actually fulfilled
by his/her final work.

The standard caveat applies:

\begin{quote}
\index{guarantee}%
This template package is provided and licensed ``as is'' without warranty
of any kind, either expressed or implied, including, but not limited to,
the implied warranties of merchantability and fitness for a particular
purpose. Yadda, yadda, yadda, \ldots
\end{quote}

In case of any problem with the use of \texttt{utdiss2}, send me email
at \url{mccluskey@mail.utexas.edu}.
\end{comment}


\chapter{Background}
\index{Background\emph{Background}}%

Mathematical Expressions MEs form an essential part of scientific
and technical documents. Mathematical Expressions can be typeset or
handwritten which uses two dimensional arrangements of symbols to
transmit information. Recognizing both form of mathematical
expressions are challenging. A variation to handwritten ME is
cursive handwriting. Unconstrained cursive property of such
handwritten expressions poses a major challenge to its recognition.

Generally speaking understanding and recognizing mathematical
expression, whether typeset or handwritten, involves three
activities: Expression localization, symbol recognition and
symbol-arrangement analysis. ME localization involves finding and
extracting mathematical expression from the document. Symbol
recognition converts the extracted expression image into a set of
symbols and symbol arrangement analyzes the spatial arrangement of
set of symbols to recover the information content of the given
mathematical notations.

Now based on the recognition process, symbol recognition activity
can further subdivided as 1) preprocessing - noise reduction,
deskewing, slant correction etc, 2) segmentation to isolate symbols
3) and finally, recognition. Similarly depending upon the
symbol-arrangement algorithm, symbol arrangement analysis can be
further subdivided into a) identification of spatial relationships
among symbols b) identification of logical relationships among
symbols 3) construction of meaning. These processes can be executed
in series or in parallel with latter processes providing contextual
feedback for the earlier processes. The order of these recognition
activities can vary somewhat, for example, partial identification of
spatial and logical relationships can be performed prior to symbol
recognition.

\section{Preprocessing}
\index{Preprocessing@\emph{Preprocessing}}%

Preprocessing is required to eliminate irregularities and noise from
the image, especially in handwritten character recognition. Certain
preprocessing method requirements may depend upon the techniques
used for recognition. \cite{gyeonghwan1997lda} uses chain code
method for handwritten image representation. Preprocessing involves
slant angle correction in which global slant angle from different
vertical lines is estimated and tangent of the estimated global
slant angle is used to correct for slant. Smoothing of image
involves elimination of small blobs (noise) on the contour. A
sliding 3-component one dimensional window is applied overall
components during which components are removed or added based on the
orientation of components. Average stroke width is estimated by
dividing chain code contours horizontally and by tracing left to
right various distances between outer and inner contour.
\cite{jcai1999issi} performs size normalization to reduce variation
in character size. To avoid significant deformation due to directly
scaling of all images to identical size, a holistic approach is used
for scaling in which if width/height ratio is less than 0.8 then
scale is identical horizontally and vertically otherwise the scale
factor is set to 0.8 to prevent large variation in image width.

\section{Character segmentation}
\index{Character segmentation@\emph{Character segmentation}}%

Character segmentation, next step in ME recognition, has long been a
critical area of OCR process. Depending upon the requirement,
character segmentation techniques is divided into four major
headings \cite{CaseyLecolinet1996}. Classical approach of
segmentation also called dissection technique consists of
partitioning the input image into sub-images based on their inherent
features, which are then classified. Another approach to
segmentation is a group of techniques that avoids dissection and
segments to image either explicitly by classification of
pre-specified windows, or implicitly by classification of subsets of
spatial features collected from the image as a whole. Another
approach is a hybrid approach employing dissection but using
classification to select from admissible segmentation possibility.
Finally holistic approach avoids segmentation process itself and
performs recognition entire character strings.

Various techniques have been used for segmentation that involves
dissection. White spaces between the characters are used to detect
segmentation points. Pitch which is the number of characters per
unit of horizontal distance provides a basis for estimating
segmentation points. The segmentation points obtained for a given
line should be approximately equally spaced at the distance that
corresponds to pitch \cite{CaseyLecolinet1996}.

Inter-character boundaries can be obtained if most segmentation
takes place by finding columns of white. Now all segmentation points
that do not lie near these boundaries can be rejected as caused due
to broken characters. Similarly we can estimate missed points due to
merged characters. Hoffman and McCullough gave a framework for
segmentation that involves three steps i.e. 1) Detection of the
start of the character, 2) A decision to begin testing for the end
of a character called sectioning, 3) Detection of end-of-character.
Sectioning is done by weighted analysis of horizontal black runs
completed versus run still incomplete.  Once sectioning determines
the regions of segmentation, rules were invoked to segment based on
either an increase in bit density or the use of special features
designed to detect end-of-character.

In \cite{arica2002ocr}, segmentation in cursive handwritten
characters is performed in the binary word image by using the
contour of the writing. Determination of segmentation regions is
done in three steps. In first step a straight line is drawn in the
slant angle direction from each local maximum until the top of the
word image. While going upward in the slant direction, if any
contour pixel is hit, this contour is followed until the slope of
the contour changes to the opposite direction. An abrupt change in
the slope of the contour indicates an end point. A line is drawn
from the maximum to the end point and path continues to go upward in
slant direction until the top of the word image. In step 2, a path
in the slant direction from each maximum to the lower baseline, is
drawn. Step 3 follows the same process as in step 1 in order to
determine the path from lower baseline to the bottom of the word
image. Combining all the three steps gives the segmentation regions.
In \cite{gyeonghwan1997lda} segmentation involves detecting
ligatures as segmentation points in cursive scripts. Alternatively,
concavity features in the upper contour and convexities in the lower
contour are used in conjunction with ligatures to reduce the number
of potentials segmentation points.

Another dissection technique that applies to non-cursive characters
is bounding box technique \cite{CaseyLecolinet1996}. In this
analysis, the adjacency relationships between characters are tested
to perform merging or their size or aspect ratios are calculated to
trigger splitting mechanisms.  Another involves splitting of
connected components. Connected components are merged or split
according to rules based on height and width of the bounding boxes.
Intersection of two characters can give rise to special image
features and different dissection methods have been developed to
detect these features and to use them in splitting a character
string images into sub-images.

\cite{chen2000sso} focuses on segmentation of single and multiple
touching character segmentation. \cite{chen2000sso} proposes a new
technique that links the feature points on the foreground and
background alternately to get the possible segmentation path.
Mixture Gaussian probability function is determined and used to rank
all the possible segmentation paths. Segmentation paths construction
is performed separately for single touching characters and for
multiple touching characters. All the paths from to two analysis are
collectively processed to remove useless strokes and then mixture
Gaussian probability function is applied to decide which on is the
best segmentation path.

Another kind of approach to character segmentation is recognition
based approach. In these segmentation processes letter segmentation
is a by-product of letter recognition. The basic principle is use a
mobile window of variable width to provide sequences of tentative
segmentation which are confirmed (or not) by character recognition.
A technique called Shortest Path Segmentation selects the optimal
combination of cuts from the predefined set of candidate cuts that
construct all possible legal segments through combination. A graph
whose nodes represent acceptable segments is the created. The paths
of these graphs represent all legal segmentations of the word. Each
node of the graph is then assigned a distance obtained by the neural
net recognizer. The shortest path though the graph thus corresponds
to the best recognition and segmentation of the word. An alternative
method attempts to match subgraphs of features with predefined
character prototypes. Different alternative are represented by a
directed network whose nodes correspond to the matched subgraphs.
Word recognition is done by searching for the path that gives the
best interpretation of the word features.

\section{Symbol-Arrangement Analysis}
\index{Symbol-Arrangement Analysis@\emph{Symbol-Arrangement Analysis}}%

One approach to symbol-arrangement analysis is syntactic approach.
Syntactic approach makes use of two dimensional grammar rules to
define the correct grouping of math symbols. Co-ordinate grammar for
recognition is presented by Anderson. The grammar specifies
syntactic rules that subdivide the set of symbols into several
subsets, each with its own syntactic subgoal. The final
interpretation result is given by the m attribute of the grammar
start's symbol where m represents ASCII encoding of the meaning of
symbol-set. Although coordinate grammar provides a clear and well
structured recognition approach, its slow parsing speed and
difficulty to handle errors are its major drawbacks. In [8], a
syntactic approach is adopted in which a system consisting of
hierarchy of parsers for the interpretation of 2-D mathematical
formulas is described. The ME interpreter consists of two syntactic
parser top-down and bottom-up. It starts with a priority operator in
the expression to be analyzed and tries to divide it into
sub-expressions or operands which are then analyzed in the same way
and so on. The bottom-up parser chooses from the starting character
and from the neighboring sub-expressions the corresponding rule in
the grammar. This rule gives instructions to the top-down parser to
delimit the zones of neighboring operands and operators.

Garain and Chaudhari in \cite{garain2004roh}, proposes a two pass
approach to determine arrangement of symbols. The first pass is a
scanning or lexicon analysis that performs micro-level examination
of the symbols to determine the symbol groups and to determine their
categories or descriptors. The second pass is parsing or syntax
analysis that processes the descriptors synthesized in the first
pass to determine the syntactical structure of the expression. A set
of predefined rules guides the activities in both the passes.

Another symbol-arrangement analysis approach is projection profile
cutting. It involves recursive projection-profile cutting. Cutting
by the vertical projection profile is attempted first, followed by
horizontal cuts for each resulting regions. The process repeats
until no further cutting is possible. The resulting spatial
relationships are represented by a tree structure.  Although the
method looks simple and efficient technique, it is still under study
and also involves additional processing for symbols like square
root, subscripts and superscripts as these can be handled by
projection profile cut.

Another approach discussed is the Graph Rewriting. Graph rewriting
involves information represented as an attributed graph and the
graph get updated through the application of graph-rewriting rules.
An initial graph contains one node to represent each symbol, with
nodes attributes recording the spatial coordinates of the symbol.
Graph rewriting rules are applied to add edges representing
meaningful spatial relationships. Rules are further applied to prune
or modify these edges identifying logical relationships from the
spatial relationships. In [7], Ann Grbavec and Dorothea Blostein
proposed a novel-graph rewriting techniques that addresses the
recursive structure of mathematical notations, the critical
dependence of the meaning upon operator precedence and the presence
of ambiguities that depends upon global context. The recognition
system proposed called EXPRESSO, is based on
Build-Constrain-Rank-Incorporate model where the Build phase
constructs edges to represent potentially meaningful spatial
relation- ships between symbols. The Constrain phase applies
information about the notational conventions of mathematics to
remove contra- dictions and resolve ambiguities. The Rank phase uses
information about the operator precedence to group symbols into
sub-expressions and the Incorporate phase interprets
sub-expressions.

Twaakyondo and Okamoto \cite{twaakyondo1995saa} discuss two basic
strategies to decide the layout of structure of the given
expression. One strategy is to check the local structures of the
sub-expressions using a bottom-up method (specific structure
processing). It is used to analyze nested structures like
subscripts, superscripts and root expressions. The other strategy is
to check the global structure of the whole expression by a top-down
method (fundamental structure processing). It is used to analyze the
horizontal and vertical relations between sub-expressions. The
structure of the expression is represented as a tree structure.

Chou in [11] proposed a two-dimensional stochastic context-free
grammar for recognition of printed mathematical expressions. The
recognized symbols are parsed with the grammar in which each
production rule has an associated probability. The main task of the
process is to find the most probable parse tree for the input
expression. The overall probability of a parse tree is computed by
multiplying together the probabilities for all the production rules
used in a successful parse.

\section{Conclusion}
\index{Conclusion@\emph{Conclusion}}%

As we saw through the survey, there have been tremendous advances in
the field of character recognition from so many years of research.
Some experiment tried to focus on one activity of recognition
process while other tried to build a complete system for character
recognition. Some researchers assumed complete well recognized
symbols are given and they focus on the symbol-arrangement
(structural) analysis of the recognized symbols. This survey
concentrated mainly on the two activities of character recognition
i.e. segmentation of symbols and symbol-arrangement of recognized
symbols.

Segmentation processes discussed have some limitations such as some
are restricted to be applied to cursive handwriting while other
focuses on non-cursive handwriting. Some researchers focus on
certain subset of mathematical symbols because of large mathematical
symbol set. Some concentrate on single touching characters some on
multiple touching characters. Certain approaches of segmentation
like holistic approach that recognizes entire word as a unit have
drawback of being restricted to predefined lexicons. Hence more
efficient and robust segmentation process is required as further
analysis of ME recognition depends on segmentation and recognition
of symbols.

Symbol-arrangement analysis discussed shows wide variations in
approaches. Some approach exploits the operator precedence property
of mathematical expression while some performs different level of
analysis (lexicon and syntax) to first group symbols into different
categories and then perform structural analysis using predefined
rules. Some using graph rewriting technique in which mathematical
symbols are linked to each other through graph rewriting rules. Some
use stochastic grammar rules to represent to the relationship
between symbols while some intelligently looks for local structures
of the expression to determine the features like nested, above or
below followed by global analysis to check for the correctness of
the expression as a whole and rectify wrong arrangements of symbols.
Symbol arrangement analysis may be not so crucial for problems that
involve only standard English character but problems like
recognition of mathematical expressions where the actual position
and location of symbols is important and there are many implicit
meaning to symbols which depends on their arrangement, it is
absolutely important to perform symbol arrangement analysis for
better recognition result.


\chapter{Methodology}
\index{Methodology@\emph{Methodology}}%


\chapter{Results and Discussion}
\index{Results and Discussion@\emph{Results and Discussion}}%


\chapter{Conclusion and Future Work}
\index{Conclusion and Future Work@\emph{Conclusion and Future Work}}%


%\include{chapter-instructions}

%\include{chapter-howtouse}

%\include{chapter-makingbib}

%\include{chapter-tables+figs}

%\include{chapter-math}


%%%%%%%%%%%%%%%%%%%%%%%%%%%%%%%%%%%%%%%%%%%%%%%%%%%%%%%%%%%%%%%%%%%%%%
% Appendix/Appendices                                                %
%%%%%%%%%%%%%%%%%%%%%%%%%%%%%%%%%%%%%%%%%%%%%%%%%%%%%%%%%%%%%%%%%%%%%%
%
% If you have only one appendix, use the command \appendix instead
% of \appendices.
%
%\appendices
%\index{Appendices@\emph{Appendices}}%

%\chapter{Lerma's Appendix}
\index{Appendix!Lerma's Appendix@\emph{Lerma's Appendix}}%
The source \LaTeX{} file for this document is no longer quoted in
its entirety in the output document. A \LaTeX{} file can 
include its own source by using the command
\cn{verbatiminput\{\cn{jobname}\}}.



%%%%%%%%%%%%%%%%%%%%%%%%%%%%%%%%%%%%%%%%%%%
\chapter{My Appendix \#2}
\index{Appendix!My Appendix \#2@\emph{My Appendix \#2}}%
\section{The First Section}
This is the first section.
This is the second appendix.

\section{The Second Section}
This is the second section of the second appendix.

\subsection{The First Subsection of the Second Section}
This is the first subsection of the second section of the second appendix.

\subsection{The Second Subsection of the Second Section}
This is the second subsection of the second section of the second appendix.

\subsubsection{The First Subsubsection of the Second Subsection of
		the Second Section}
This is the first subsubsection of the second subsection of the
second section of the second appendix.

\subsubsection{The Second Subsubsection of the Second Subsection
		of the Second Section}
This is the second subsubsection of the second subsection of the
second section of the second appendix.


%%%%%%%%%%%%%%%%%%%%%%%%%%%%%%%%%%%%%%%%%%%
\chapter{My Appendix \#3}
\index{Appendix!My Appendix \#3@\emph{My Appendix \#3}}%

\section{The First Section}
This is the first section.
This is the third appendix.

\section{The Second Section}
This is the second section of the third appendix.





%%%%%%%%%%%%%%%%%%%%%%%%%%%%%%%%%%%%%%%%%%%%%%%%%%%%%%%%%%%%%%%%%%%%%%
% Generate the bibliography.                         %
%%%%%%%%%%%%%%%%%%%%%%%%%%%%%%%%%%%%%%%%%%%%%%%%%%%%%%%%%%%%%%%%%%%%%%
%                                    %
% NOTE: For master's theses and reports, NOTHING is permitted to     %
%   come between the bibliography and the vita. The command      %
%   to generate the index (if used) MUST be moved to before      %
%   this section.                            %
%                                    %
%
\nocite{*}      % This command causes all items in the           %
                % bibliographic database to be added to          %
                % the bibliography, even if they are not         %
                % explicitly cited in the text.              %
        %                            %
  % Here the bibliography           %
        % is inserted.                %

\index{Bibliography@\emph{Bibliography}}%
\bibliographystyle{plain}
\bibliography{diss}
%%%%%%%%%%%%%%%%%%%%%%%%%%%%%%%%%%%%%%%%%%%%%%%%%%%%%%%%%%%%%%%%%%%%%%


%%%%%%%%%%%%%%%%%%%%%%%%%%%%%%%%%%%%%%%%%%%%%%%%%%%%%%%%%%%%%%%%%%%%%%
% Generate the index.                            %
%%%%%%%%%%%%%%%%%%%%%%%%%%%%%%%%%%%%%%%%%%%%%%%%%%%%%%%%%%%%%%%%%%%%%%
%                                    %
% NOTE: For master's theses and reports, NOTHING is permitted to     %
%   come between the bibliography and the vita. This section     %
%   to generate the index (if used) MUST be moved to before      %
%   the bibliography section.                    %
%                                    %
%\printindex%    % Include the index here. Comment out this line      %
%       % with a percent sign if you do not want an index.   %
%%%%%%%%%%%%%%%%%%%%%%%%%%%%%%%%%%%%%%%%%%%%%%%%%%%%%%%%%%%%%%%%%%%%%%


%%%%%%%%%%%%%%%%%%%%%%%%%%%%%%%%%%%%%%%%%%%%%%%%%%%%%%%%%%%%%%%%%%%%%%
% Vita page.                                 %
%%%%%%%%%%%%%%%%%%%%%%%%%%%%%%%%%%%%%%%%%%%%%%%%%%%%%%%%%%%%%%%%%%%%%%
\begin{vita}
Amit Arun Pillay was born in Mumbai, India on September 21, 1984,
the son of Arun Pillay and Uma Pillay. He received the Bachelor of
Engineering degree in Information Technology from Veermata Jijabai
Technological Institute, Mumbai, India in 2006. He is currently
pursuing his Master of Science degree from Rochester Institute of
Technology, United States of America. His research interest includes
Pattern Recognition, Computer Vision, Image Processing and Machine
Learning. His current research includes combining syntactical
pattern recognition techniques.

\end{vita}
\end{document}
