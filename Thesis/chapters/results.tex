\chapter{Results and Discussion}
\index{Results and Discussion@\emph{Results and Discussion}}%
\label{chap:results}

\section{Evaluation}\label{sec:results-evaluation}

\todo[inline]{Evaluation of each scheduler will consist of complete testing:\\
- Run $PTree_N$ for $1 \leq N \leq P+1$ where $P$ is number of processors.\\
- Run $PRing_N$ for $N \in \{ 1, P, B-1, B, 2*B \}$ where $B$ is max batch size where applicable.\\
- Run $ClusterComm_{(N,M)}$ for $N \in \{ 2, 4, ..., P, B, 2*B \}$ and $ 1 \leq M \leq \lfloor N / 2 \rfloor$ \\
- Run $ChugMachine_N$ for $1 \leq N \leq P+2$  \\
- Run $JumpShip_{(W,H)}$ for $2 \leq W \leq P+1$ and $1 \leq H \leq 3$ 
}

\todo[inline]{Composing simulations may lead to interesting results though. 
I would like to test, at least: \\

- $UserInput$ and $ChugMachine$ to examine Interactivity.\\
- $PTree$ and $ChugMachine$ or $ClusterComm$ to have both structured subcomponents, and otherwise.
}

\subsection{Classical Schedulers}\label{sec:results-evaluation-classical}

\todo[inline]{ There are 5 schedulers, STRR, STDQ, MTRRGQ, MTRRWS-SQ, MTRRWS-IS which 
need to be evaluated. However, I wont know what's interesting until after all tests have
been run.}

\subsection{Cooperativity Feedback Schedulers}\label{sec:results-evaluation-feedback}

\todo[inline]{ There are 3 schedulers, Longevity-Batcher, Channel-Pinner, Graph Shuffling.
I will primarily be looking at:\\

- How does Longevity Batcher degrade? Does it turn into MTRRWS?\\
- How quickly and thoroughly can Channel-Pinner saturate the cores with work?\\
- How does Graph Shuffling handle large and small process queues?
}


\section{Comparisons}\label{sec:results-comparisons}

\todo[inline]{ Including the above, I will want to compare:\\
- MTRRWS-SQ and MTRRWS-IS \\
- STRR and MTRRGQ \\
- Long-Batcher and Channel-Pinner in terms of work-stealing \\
- All feedback schedulers in terms of: execution time (tick count), saturation, evenness of work-load.
}

\subsection{Channel Implementations}\label{sec:results-channel-implementations}

\todo[inline]{ There are 3 mechanisms provided by ErLam which modify the channel
  implementation: Blocking Channels, Absorption Channels, and then channel pinning
based on a spreading algorithm. In this section I'd primarily like to look at the
first two.
}

\section{A Comment on Swap Channels}\label{sec:results-swap-channels}

Swap channels provided a number of benefits on the side of the language 
designer. They reduce the complexity of channel implementation, and as shown, 
they lend themselves to a number of possible designs. We demonstrated just two 
possible implementations, the Blocking and Absorption channels. As such, the 
concept of a swap channel as a language primitive is extremely attractive. 
However, swapping poses some problems for realistic applications which we would
now like to discuss.

\begin{figure}
\centering
\inputminted[frame=lines,fontsize=\footnotesize]{csharp}{code/badclustercomm.els}
\caption{A naive but ineffectual $ClusterComm_{(N,M)}$ implementation.} 
\label{fig:bad-clustercomm}
\end{figure}

First, a swap channel does not lend itself to a level of fairness that would be
expected by a programmer. We lead with our implementation of 
$ClusterComm_{(N,M)}$ as example, which due to being on top of swap channels, 
was made to be much more enigmatic. Figure~\ref{fig:bad-clustercomm} gives an
example implementation of $ClusterComm_{(N,M)}$. Note the third parameter to
the application, which denotes the number of time's each of the $N$ processes
should communicate. The system therefore blocks until all processes synchronize
$X$ times before quiting.

Due to this, we must pose several restrictions on the possible values of $N$ and
$M$. Namely, $N$ must be an even number, since all processes would need to have a 
partner to swap with, and $M$ must be no greater than $\lfloor N/2 \rfloor$, any
more and it would be possible for a process to hang indefinitely.

However, these two restrictions are not enough to guarantee the process terminates.
In fact, most runs of this application, with any value of $N > 2$ would most 
likely hang forever. This is due to the bias our program creates when spawning
processes, as well as the type of fairness the swap channel semantics provides.

First, our bias we introduce is merely because we cannot batch spawn a set of
processes at the same time. As such we will spawn one process at a time which
may get a chance to run before all others. As such the first several processes
may reach their synchronization limit before we are even done spawning the rest
of the processes. Due to this, we may have a case where all but $M$ processes
have completed, and thus all channels are blocked indefinitely.

Secondly, the channel semantics have no inherent preference for unseen 
or new processes. The scheduler may easily get in a loop of running the same
subset of processes repeatedly, this would have the same effect as the above
even if we were able to solve the bias problem. As such, this is inherently an
issue with the capabilities of swap channels. 

Thus, the best we can do for $ClusterComm_{(N,M)}$, is to run until at least 
$N-M$ processes have met their quota. Note this approach is only acceptable 
under Symmetric message passing constructs. In asymmetrical, even if there was
a guarantee of an equal number of senders and receivers, all senders could be
blocked on one channel while all receivers could be blocked on another. 

But this issue points to another problem with swap channels, insofar as they do
not lend themselves to being primitives at all. Due to this fairness issue, a 
language with swap channels would be unable to build the asymmetrical constructs
most user's would like. As such, they have been useful merely for simulation 
purposes.

