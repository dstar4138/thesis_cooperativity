\chapter{Conclusion and Future Work}
\index{Conclusion and Future Work@\emph{Conclusion and Future Work}}%
\label{chap:conclusions}

\section{The ErLam Toolkit}\label{sec:conclusions-erlam}


\section{Effectiveness of Cooperativity as a Metric}
\label{sec:conclusions-cooperativity}



\section{Future Work}\label{sec:future-work}

There are a number of appealing avenues of improvement for the ErLam toolkit. The
report generation mechanism could be extended and tied into the logging system a 
bit more  closely. For example, the implementation of a real-time viewer would be an 
interesting extension. There are obviously a larger number of metrics which may lead
to better cooperativity classifications as well. It may be more fruitful, for 
example, to keep track of communication partners rather than channel names. 
The core language is also an appealing simulation implementation language, as such a 
larger library of testing primitives would benefit the language designer community 
greatly. Furthermore a complete catalog of parameterized executions would also aid 
in analysis and scheduler comparison.

In terms of cooperativity as a feedback metric, it would be interesting to further
tweak the three cooperativity-conscious schedulers already implemented. Perhaps 
composing the scheduler mechanics themselves may lead to a more robust algorithm.
For example, the combination of the Channel Pinning scheduler and the 
Bipartite-Graph Aided scheduler may complement each other. However, in short, 
the future of cooperativity as a feedback metric looks promising.

