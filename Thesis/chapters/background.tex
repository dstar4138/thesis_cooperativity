\chapter{Background}
\index{Background\emph{Background}}%
\label{chap:background}

%% TODO SECTIONS: It would be good to rework the below to fit into the following
%%  primary sections. We can have subsections for each
%\section{Classical Scheduling}\label{sec:classical scheduling}
%\section{Feedback-Enabled Scheduling}\label{sec:feedback systems}
%\section{Message Passing}\label{sec:message passing}


Since the formalization of feedback driven systems and the advent of Cybernetics, multiple fields have attempted 
to mold these principles to their own models; and run-time schedulers are no exception. This is due, in part, from
process scheduling in parallel systems being fundamentally an NP-Complete problem \cite{bruno1976computer}. Instead,
focus has been more fruitful when pursuing the optimization of various metrics using some particular objective 
function \cite{garey1978performance} to tune for particular edge cases. As such, scheduling based on feedback 
metrics is not new \cite{dietz1997use}.

There is a big distinction though, which can be made between the effects of control theory in classical cybernetic applications 
versus that of run-time systems. This is primarily in the adaptation of the controller in the generic feedback 
loop (figure~\ref{fig:controller}).
In typical mechanical feedback loops there are two scenarios which need to be avoided: resonance 
and rapid compensation. It can be seen that most controller models will attempt to damp the adjustments to reduce 
oscillation which could cause resonance or sharp spikes in behavior based on its output. This is due to the  
limitations of the physical space in which they are having to deal with. 

However, in run-time scheduling systems we would very much like to do the opposite. We would prefer tight oscillations or 
consistent behavior of our runtime so as to achieve minimal overhead from our modifications. We can also compensate, 
to reach our reference signal, as quickly as we need to as there are no physical restrictions for our modifications. 

\begin{figure}[htp]
\centering
\includegraphics[scale=0.55]{\detokenize{Feedback_loop_with_descriptions}} %detoken because of '_' char.
%\includegraphics{\detokenize{Feedback_loop_with_descriptions}} %detoken because of '_' char.
\caption{A classical feedback loop representation.}
\label{fig:controller}
\end{figure}
\todo{Not sure why the image is blurry}

Also, note that run-time systems tend to extend to more than just the process-scheduling systems. They also encompass  
garbage collection, dynamic type checking, statistical collection and other debugging mechanisms, and general 
resource allocation (\eg~I/O). For the purposes of this paper we will be exclusively talking about the process
scheduling system and intentionally ignore the effects of these other, albeit important, sub-systems. This 
distinction must be made however, as some scheduling systems also take into account the effects of other
sub-systems on the placement and order of process evaluation. For example, White \etal~\cite{white2012automated}
discuss heap size as a metric for process selection as a means of curtailing garbage collection effects.

\section{Classical Run-Time Scheduling}

Another distinction must be made as far as the level of foresight the scheduling systems have, at least, within this paper.
There is a spectrum of clairvoyance 
\todo{Citation needed, taxonomy paper?}
in classical job-scheduling, in that on one end, job-schedulers have full foresight
over the jobs which will enter the queue and their order. These schedulers have the opportunity to optimize for future
events, which is a luxury the scheduling systems that this paper discusses do not have. 

However, as it is a spectrum, there is a single point of knowledge this subrange of schedulers can assume. Namely, 
that the first job will always be the last, and all other jobs will spawn from it. Thus there will always be a single 
process in the queue at the beginning. This is true as the runtime will always require an initial primary process 
(\eg~the `main' function), and once that function is completed, the system is terminated (despite the cases of unjoined
children). Apart from this, all other insights will need to be gleaned from the evaluation of this initial process.


\section{Message Passing}

In concurrent systems, there are a number of methods for inter-process communication. Arguably though, one 
of the more popular abstractions is the idea of message passing. Message passing in general can be broken down into 
two types, asynchronous or synchronous, and then further by how they are implemented. However, when discussing process
scheduling the method of their implementation is often of some consequence \todo{Needs citation?}.

\begin{figure}[htp]
\centering
\Tree [ .{Message Passing}
			[ .Async 
				Direct 
				Indirect 
			] 
			[ .Sync 
				Asymmetric
				Symmetric 
			]
	   ]
\caption{A High-Level Message-Passing Taxonomy}
\label{fig:mptax}
\end{figure}

In asynchronous message passing a process can either be provided a rendezvous point or an identifier for another process.
To send a message in either case requires pushing/copying the message into a shared or global memory space for another
process to access (possibly) at a later time. This push/copy can be done in a lock free manner with some lower level
atomic data structures such as a double-ended queue. But in either a locked or lock-free manner, the process performing
the send still blocks (if only for a moment) on the operation.

In terms of scheduling, a language with asynchronous message passing can ignore the effects of blocking operations, but
will need to look closer at process placement to take advantage of possible gains in cache affinity \cite{debattista2002cache}.
For example, the effects of the cache on direct message passing (\eg~a process mailbox) can be substantial as two processes
on different cores will need to copy stack frames to memory and back rather than just a pointer to the message in the 
originator's stack frame. In indirect message passing the task is even worse as multiple processes may need access to the
same data.

In synchronous message passing, a process must meet another at a provided rendezvous point but can either be symmetrical
or asymmetrical. Note that the rendezvous point is not a requirement in the sense that direct synchronous messaging isn't 
possible. Instead we think of a rendezvous point in synchronous communication to be time bound rather than location bound (\ie~two
processes are blocked until communication occurs, the implementation of this passing is irrelevant to this 
classification).

Asymmetrical message passing is synonymous with Milner's Calculus of Communicating Systems \cite{milner1982calculus} 
or standard $\Pi$-Calculus \cite{palamidessi1997comparing}, in that you have a sender and a receiver which will 
block on their respective functions around an anonymous channel until the pass has been completed. This differs from a 
symmetrical message passing in that the only operation on the channel is a blocking function which swaps values 
with the process at the other end.

Note, it is possible to simulate symmetrical message passing on asymmetrical message channels, but in terms of scheduling of
synchronizing processes, order is now a factor that needs to be considered. On top of this, directionality can also be a factor
which complicates the channel implementation. Namely, the internal queuing of senders or receivers may not percolate hints up to
the scheduler regarding their queue position. For the alternative, symmetrical message passing or swap channels, the order is
directly handled by the scheduling of the system (\ie~the order at which the channels evaluate the $swap$ command can be 
directly governed).

\section{Example Scheduling Feedback Systems}

A basic feedback scheduler can be explained using a metric like 'waiting time vs running time'. A standard round-robin
scheduler may utilize a queue of processes and attempts to be fair by giving every process an equal chance to run.
However, in a contrived example such as a fork-bomb, the scheduler may never reach the end of the queue and the rest of 
the system could starve. In our example feedback enabled system each process could be ranked according to its ratio of 
waiting time over running time. Thus as time went on, even in the event of a fork-bombing effects the beginning of the
queue would still get some time to run.

There is quite a literature behind the types of metrics used though. Yet statistical collection is 
only a third of the feedback loop. A classification of the system must be made by a control mechanism based on the 
recorded state, and then a modification must be made to the scheduler which should positively 
effect the state of the system's future. The key is to make sure this effect is transparent and results in minimal
overhead, while still achieving the objective function.

This objective function, for the purposes of simplicity is typically execution time. However, it can also take into
account resource utilization, process/channel fairness, fulfillment of real-time guarantees, \etc~to rate the 
schedule as a whole. Note that this function is an external measure (evaluation) of the schedulers effectiveness,
which will be brought up further in section~\ref{sec:workevaluation}. For now however, note that the objective 
function and control metrics are separate and evaluate two different things.

Before discussing Cooperativity as a feedback metric though, it is important to introduce previous metrics and their
respective control schemes. Interactivity is one metric which evaluates the system's current scheduler. It looks at 
the ratio of communication to computation currently taking place in the system as a whole. One scheduling system which 
utilizes this metric indirectly is in CML \cite{reppy1993concurrent}.

It's method by which it grades the system is the recognition of {\em computation}-bound processes versus
{\em communication}-bound processes. A {\em communication}-bound process is one which has previous blocked itself
without using it's entire scheduled quantum, the rational for this is only a synchronizing thread would be willing
to block; thus a {\em computation}-bound process is the opposite (\ie~one that always uses its full quantum). After
labeling the processes the scheduler pushes {\em computation}-bound processes to a secondary queue so they don't eat
up all of the time. This improves the time the scheduler can spend attending the processes which need to synchronize,
which in turn improves the Interactivity of the application (\ie~GUI widgets will finish their message passing while 
the back end computes).

\section{Cooperativity as a Metric}

\todo[inline]{This section feels off, I wish I had referenced a "generic" definition by which to base all assumptions
and initial observations off of.}

As mentioned, note that the Interactivity metric looks only at the ratio of {\em communication}-bound processes to {\em computation}-bound
processes and misses the quantity of communication on a per-process level which could indicate needed changes in
scheduling quantum or process locality with those which are frequently in touch. 
This hints at a need to look for some level of cooperation between processes in the phases of an application's
life-time. 

Rather than a ratio of {\em communication} to {\em computation}, instead we need to look at a 
frequency of communication and between which processes (\ie~rate at which a process communicates with a particular
channel). One method this can be achieved by is, in addition to recording which processes are {\em communication} and 
{\em computation}-bound, the scheduler can also tag the process with it's frequency of communication on each 
channel it touches over time. By doing so, the scheduler can calculate a process's cooperativity with one or more
processes by comparing frequency counts on particular channels, perhaps by a sum of ratios.

Cooperativity, in the general sense, has been a probabilistic interpretation of how two actors can be affected by the actions of
others. This has been a sociological term as well as a biological one. Abeliovich \cite{abeliovich2005empirical}, thusly defines a
notion of positive cooperativity as: 
\begin{quote}
A process involving multiple identical incremental steps, in which intermediate states are statistically underrepresented relative to a hypothetical standard system [...] where the steps occur independently of each other. Likewise, a definition of negative cooperativity would be a process involving multiple identical incremental steps, in which the intermediate states are overrepresented relative to a hypothetical standard state in which individual steps occur independently.
\end{quote}

In other words, while a process can be cooperative, a system can be shown to possess {\em positive} or {\em negative}
cooperativity as the set of cooperative processes increases or decreases (respectively). This lends itself quite 
readily as a metric for an evolving system of phases. Assuming the cooperativity can be directly probed we can check 
it against an evolving standard state the scheduler proposes. The feedback mechanism would be to attempt to raise or
lower the cooperativity of the system to get it back to this standard state. It can do so by increasing or decreasing
the synchronization allowance each scheduler gets per a given time frame for example. This would have the effect of 
bounding the frequencies of communication.


