\chapter{Background}
\index{Background\emph{Background}}%

Mathematical Expressions MEs form an essential part of scientific
and technical documents. Mathematical Expressions can be typeset or
handwritten which uses two dimensional arrangements of symbols to
transmit information. Recognizing both form of mathematical
expressions are challenging. A variation to handwritten ME is
cursive handwriting. Unconstrained cursive property of such
handwritten expressions poses a major challenge to its recognition.

Generally speaking understanding and recognizing mathematical
expression, whether typeset or handwritten, involves three
activities: Expression localization, symbol recognition and
symbol-arrangement analysis. ME localization involves finding and
extracting mathematical expression from the document. Symbol
recognition converts the extracted expression image into a set of
symbols and symbol arrangement analyzes the spatial arrangement of
set of symbols to recover the information content of the given
mathematical notations.

Now based on the recognition process, symbol recognition activity
can further subdivided as 1) preprocessing - noise reduction,
deskewing, slant correction etc, 2) segmentation to isolate symbols
3) and finally, recognition. Similarly depending upon the
symbol-arrangement algorithm, symbol arrangement analysis can be
further subdivided into a) identification of spatial relationships
among symbols b) identification of logical relationships among
symbols 3) construction of meaning. These processes can be executed
in series or in parallel with latter processes providing contextual
feedback for the earlier processes. The order of these recognition
activities can vary somewhat, for example, partial identification of
spatial and logical relationships can be performed prior to symbol
recognition.

\section{Preprocessing}
\index{Preprocessing@\emph{Preprocessing}}%

Preprocessing is required to eliminate irregularities and noise from
the image, especially in handwritten character recognition. Certain
preprocessing method requirements may depend upon the techniques
used for recognition. \cite{gyeonghwan1997lda} uses chain code
method for handwritten image representation. Preprocessing involves
slant angle correction in which global slant angle from different
vertical lines is estimated and tangent of the estimated global
slant angle is used to correct for slant. Smoothing of image
involves elimination of small blobs (noise) on the contour. A
sliding 3-component one dimensional window is applied overall
components during which components are removed or added based on the
orientation of components. Average stroke width is estimated by
dividing chain code contours horizontally and by tracing left to
right various distances between outer and inner contour.
\cite{jcai1999issi} performs size normalization to reduce variation
in character size. To avoid significant deformation due to directly
scaling of all images to identical size, a holistic approach is used
for scaling in which if width/height ratio is less than 0.8 then
scale is identical horizontally and vertically otherwise the scale
factor is set to 0.8 to prevent large variation in image width.

\section{Character segmentation}
\index{Character segmentation@\emph{Character segmentation}}%

Character segmentation, next step in ME recognition, has long been a
critical area of OCR process. Depending upon the requirement,
character segmentation techniques is divided into four major
headings \cite{CaseyLecolinet1996}. Classical approach of
segmentation also called dissection technique consists of
partitioning the input image into sub-images based on their inherent
features, which are then classified. Another approach to
segmentation is a group of techniques that avoids dissection and
segments to image either explicitly by classification of
pre-specified windows, or implicitly by classification of subsets of
spatial features collected from the image as a whole. Another
approach is a hybrid approach employing dissection but using
classification to select from admissible segmentation possibility.
Finally holistic approach avoids segmentation process itself and
performs recognition entire character strings.

Various techniques have been used for segmentation that involves
dissection. White spaces between the characters are used to detect
segmentation points. Pitch which is the number of characters per
unit of horizontal distance provides a basis for estimating
segmentation points. The segmentation points obtained for a given
line should be approximately equally spaced at the distance that
corresponds to pitch \cite{CaseyLecolinet1996}.

Inter-character boundaries can be obtained if most segmentation
takes place by finding columns of white. Now all segmentation points
that do not lie near these boundaries can be rejected as caused due
to broken characters. Similarly we can estimate missed points due to
merged characters. Hoffman and McCullough gave a framework for
segmentation that involves three steps i.e. 1) Detection of the
start of the character, 2) A decision to begin testing for the end
of a character called sectioning, 3) Detection of end-of-character.
Sectioning is done by weighted analysis of horizontal black runs
completed versus run still incomplete.  Once sectioning determines
the regions of segmentation, rules were invoked to segment based on
either an increase in bit density or the use of special features
designed to detect end-of-character.

In \cite{arica2002ocr}, segmentation in cursive handwritten
characters is performed in the binary word image by using the
contour of the writing. Determination of segmentation regions is
done in three steps. In first step a straight line is drawn in the
slant angle direction from each local maximum until the top of the
word image. While going upward in the slant direction, if any
contour pixel is hit, this contour is followed until the slope of
the contour changes to the opposite direction. An abrupt change in
the slope of the contour indicates an end point. A line is drawn
from the maximum to the end point and path continues to go upward in
slant direction until the top of the word image. In step 2, a path
in the slant direction from each maximum to the lower baseline, is
drawn. Step 3 follows the same process as in step 1 in order to
determine the path from lower baseline to the bottom of the word
image. Combining all the three steps gives the segmentation regions.
In \cite{gyeonghwan1997lda} segmentation involves detecting
ligatures as segmentation points in cursive scripts. Alternatively,
concavity features in the upper contour and convexities in the lower
contour are used in conjunction with ligatures to reduce the number
of potentials segmentation points.

Another dissection technique that applies to non-cursive characters
is bounding box technique \cite{CaseyLecolinet1996}. In this
analysis, the adjacency relationships between characters are tested
to perform merging or their size or aspect ratios are calculated to
trigger splitting mechanisms.  Another involves splitting of
connected components. Connected components are merged or split
according to rules based on height and width of the bounding boxes.
Intersection of two characters can give rise to special image
features and different dissection methods have been developed to
detect these features and to use them in splitting a character
string images into sub-images.

\cite{chen2000sso} focuses on segmentation of single and multiple
touching character segmentation. \cite{chen2000sso} proposes a new
technique that links the feature points on the foreground and
background alternately to get the possible segmentation path.
Mixture Gaussian probability function is determined and used to rank
all the possible segmentation paths. Segmentation paths construction
is performed separately for single touching characters and for
multiple touching characters. All the paths from to two analysis are
collectively processed to remove useless strokes and then mixture
Gaussian probability function is applied to decide which on is the
best segmentation path.

Another kind of approach to character segmentation is recognition
based approach. In these segmentation processes letter segmentation
is a by-product of letter recognition. The basic principle is use a
mobile window of variable width to provide sequences of tentative
segmentation which are confirmed (or not) by character recognition.
A technique called Shortest Path Segmentation selects the optimal
combination of cuts from the predefined set of candidate cuts that
construct all possible legal segments through combination. A graph
whose nodes represent acceptable segments is the created. The paths
of these graphs represent all legal segmentations of the word. Each
node of the graph is then assigned a distance obtained by the neural
net recognizer. The shortest path though the graph thus corresponds
to the best recognition and segmentation of the word. An alternative
method attempts to match subgraphs of features with predefined
character prototypes. Different alternative are represented by a
directed network whose nodes correspond to the matched subgraphs.
Word recognition is done by searching for the path that gives the
best interpretation of the word features.

\section{Symbol-Arrangement Analysis}
\index{Symbol-Arrangement Analysis@\emph{Symbol-Arrangement Analysis}}%

One approach to symbol-arrangement analysis is syntactic approach.
Syntactic approach makes use of two dimensional grammar rules to
define the correct grouping of math symbols. Co-ordinate grammar for
recognition is presented by Anderson. The grammar specifies
syntactic rules that subdivide the set of symbols into several
subsets, each with its own syntactic subgoal. The final
interpretation result is given by the m attribute of the grammar
start's symbol where m represents ASCII encoding of the meaning of
symbol-set. Although coordinate grammar provides a clear and well
structured recognition approach, its slow parsing speed and
difficulty to handle errors are its major drawbacks. In [8], a
syntactic approach is adopted in which a system consisting of
hierarchy of parsers for the interpretation of 2-D mathematical
formulas is described. The ME interpreter consists of two syntactic
parser top-down and bottom-up. It starts with a priority operator in
the expression to be analyzed and tries to divide it into
sub-expressions or operands which are then analyzed in the same way
and so on. The bottom-up parser chooses from the starting character
and from the neighboring sub-expressions the corresponding rule in
the grammar. This rule gives instructions to the top-down parser to
delimit the zones of neighboring operands and operators.

Garain and Chaudhari in \cite{garain2004roh}, proposes a two pass
approach to determine arrangement of symbols. The first pass is a
scanning or lexicon analysis that performs micro-level examination
of the symbols to determine the symbol groups and to determine their
categories or descriptors. The second pass is parsing or syntax
analysis that processes the descriptors synthesized in the first
pass to determine the syntactical structure of the expression. A set
of predefined rules guides the activities in both the passes.

Another symbol-arrangement analysis approach is projection profile
cutting. It involves recursive projection-profile cutting. Cutting
by the vertical projection profile is attempted first, followed by
horizontal cuts for each resulting regions. The process repeats
until no further cutting is possible. The resulting spatial
relationships are represented by a tree structure.  Although the
method looks simple and efficient technique, it is still under study
and also involves additional processing for symbols like square
root, subscripts and superscripts as these can be handled by
projection profile cut.

Another approach discussed is the Graph Rewriting. Graph rewriting
involves information represented as an attributed graph and the
graph get updated through the application of graph-rewriting rules.
An initial graph contains one node to represent each symbol, with
nodes attributes recording the spatial coordinates of the symbol.
Graph rewriting rules are applied to add edges representing
meaningful spatial relationships. Rules are further applied to prune
or modify these edges identifying logical relationships from the
spatial relationships. In [7], Ann Grbavec and Dorothea Blostein
proposed a novel-graph rewriting techniques that addresses the
recursive structure of mathematical notations, the critical
dependence of the meaning upon operator precedence and the presence
of ambiguities that depends upon global context. The recognition
system proposed called EXPRESSO, is based on
Build-Constrain-Rank-Incorporate model where the Build phase
constructs edges to represent potentially meaningful spatial
relation- ships between symbols. The Constrain phase applies
information about the notational conventions of mathematics to
remove contra- dictions and resolve ambiguities. The Rank phase uses
information about the operator precedence to group symbols into
sub-expressions and the Incorporate phase interprets
sub-expressions.

Twaakyondo and Okamoto \cite{twaakyondo1995saa} discuss two basic
strategies to decide the layout of structure of the given
expression. One strategy is to check the local structures of the
sub-expressions using a bottom-up method (specific structure
processing). It is used to analyze nested structures like
subscripts, superscripts and root expressions. The other strategy is
to check the global structure of the whole expression by a top-down
method (fundamental structure processing). It is used to analyze the
horizontal and vertical relations between sub-expressions. The
structure of the expression is represented as a tree structure.

Chou in [11] proposed a two-dimensional stochastic context-free
grammar for recognition of printed mathematical expressions. The
recognized symbols are parsed with the grammar in which each
production rule has an associated probability. The main task of the
process is to find the most probable parse tree for the input
expression. The overall probability of a parse tree is computed by
multiplying together the probabilities for all the production rules
used in a successful parse.

\section{Conclusion}
\index{Conclusion@\emph{Conclusion}}%

As we saw through the survey, there have been tremendous advances in
the field of character recognition from so many years of research.
Some experiment tried to focus on one activity of recognition
process while other tried to build a complete system for character
recognition. Some researchers assumed complete well recognized
symbols are given and they focus on the symbol-arrangement
(structural) analysis of the recognized symbols. This survey
concentrated mainly on the two activities of character recognition
i.e. segmentation of symbols and symbol-arrangement of recognized
symbols.

Segmentation processes discussed have some limitations such as some
are restricted to be applied to cursive handwriting while other
focuses on non-cursive handwriting. Some researchers focus on
certain subset of mathematical symbols because of large mathematical
symbol set. Some concentrate on single touching characters some on
multiple touching characters. Certain approaches of segmentation
like holistic approach that recognizes entire word as a unit have
drawback of being restricted to predefined lexicons. Hence more
efficient and robust segmentation process is required as further
analysis of ME recognition depends on segmentation and recognition
of symbols.

Symbol-arrangement analysis discussed shows wide variations in
approaches. Some approach exploits the operator precedence property
of mathematical expression while some performs different level of
analysis (lexicon and syntax) to first group symbols into different
categories and then perform structural analysis using predefined
rules. Some using graph rewriting technique in which mathematical
symbols are linked to each other through graph rewriting rules. Some
use stochastic grammar rules to represent to the relationship
between symbols while some intelligently looks for local structures
of the expression to determine the features like nested, above or
below followed by global analysis to check for the correctness of
the expression as a whole and rectify wrong arrangements of symbols.
Symbol arrangement analysis may be not so crucial for problems that
involve only standard English character but problems like
recognition of mathematical expressions where the actual position
and location of symbols is important and there are many implicit
meaning to symbols which depends on their arrangement, it is
absolutely important to perform symbol arrangement analysis for
better recognition result.
